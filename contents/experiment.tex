\chapter{实验参数和流程}
\section{5分钟语法参考}

{\kaishu 要流畅使用\PRCthesis 需要用户对\LaTeX 以及\textsc{Bib}\TeX 有一定了解,下面这个语法参考只能起到抛砖引玉的作用。如果你从来没有接触过\LaTeX 或者\XeTeX ,建议先学习相关知识,磨刀不误砍柴工。}
\medskip
\begin{itemize}
	\item \LaTeX 源文件中,主要有三种元素:你的文字,命令,以及环境;
	\item 直接输入即可你想要写的文字即可,对于英文,文字间多于一个的空格都会转为一个空格;
	\item 如果你想开启一个新的自然段,请在写新内容前空一个(或多个)全空的行;
	\item \LaTeX 的命令{\heiti 全部}都以\cs{ }开头,例如\cs{XeTeX}可以生成\XeTeX ;
	\item 有的命令{\heiti 必须}带参数,比如\cs{zihao\{-4\}}可以将命令之后的内容的字号调整为小四;
	\item 有的命令能带可选参数,例如\cs{usingpackage\{metalogo\}}可以载入\pkg{metalogo}宏包;
	\item 宏包中有宏包作者自己定义的命令,能够让你更容易地完成某些事情,比如\pkg{mhchem}能够引入让你方便地表示化学式的命令\cs{ce};
	\item \LaTeX 的源代码主要分为两个部分,导言部分和文档部分。其中,文档部分以\cs{begin\{document\}}开头,以\cs{end\{document\}}结尾,只有在这个范围内你才能完成排版工作;
	\item \LaTeX 对(简单或复杂的)数学式的支持是其一大亮点,数学环境使用\texttt{\${ }\$}包裹;
	\item 环境由\csgo{begin}{环境名}开头,以\csgo{end}{环境名}结尾,是的,文档部分是一个巨大的环境;
	\item 报错说没有这个命令?检查是否载入了必要的宏包,再检查命令后面是否直接跟随了汉字,在它们之间加个空格就好;
	\item \LaTeX 是一门语言,新手经常会遇到无法编译通过的语法错误,这时建议你仔细检查花括号是否平衡,命令是否敲错,参数数目和类型是否正确,如果还是不行,可以在网络上搜索一番或者问问同事。
	\item 命令之间或者之内的空格和缩进以及回车不是必须的,事实上没有它们\LaTeX 也可以正常工作,但是代码的可读性就会大打折扣了;
	\item 对了,使用\texttt{\%}来开启一个行注释,注释的内容不参与编译,你可以在这里写下自己的小秘密;
	\item 有质量的国内\TeX 社区是\textsc{CTeX}社区,更有质量的国外的是\textsf{StackExchange};
	\item \TeX\textsc{Studio}是一个很棒的\LaTeX 编辑环境,推荐你尝试一番。
\end{itemize}

\section{查询文档}
在你对宏包或者环境包有疑问的时候,可以再命令行中输入:

\texttt{texdoc 宏包或环境名称}

回车后对应的用户文档会自动打开。





